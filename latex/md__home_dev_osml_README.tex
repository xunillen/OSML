\href{https://www.codefactor.io/repository/github/nullcharmander/osml}{\texttt{ }}

C\# and C based library made for easy media management.

\subsection*{Notes}


\begin{DoxyItemize}
\item This library is in experimental phase, and many things may break.
\item \mbox{\hyperlink{namespaceOSML}{O\+S\+ML}} was not tested on windows, so Linux is only supported platform(for now).
\item C code needs to be rewritten.
\end{DoxyItemize}

\subsection*{Feature list}

this is work in progress

\subsubsection*{Main}


\begin{DoxyItemize}
\item \mbox{[}x\mbox{]} Media caching using X\+ML and S\+Q\+Lite
\item \mbox{[} \mbox{]} Log support(\+Osml\+Log)
\item \mbox{[} \mbox{]} Settings storage support
\end{DoxyItemize}

\subsubsection*{Audio}


\begin{DoxyItemize}
\item \mbox{[}x\mbox{]} Main Audio support
\item \mbox{[}x\mbox{]} D3\+V2 metadata support
\item \mbox{[}x\mbox{]} I\+D3\+V1 metadata support
\item \mbox{[}x\mbox{]} Albums support
\item \mbox{[}x\mbox{]} Artist support
\item \mbox{[} \mbox{]} Playlist support
\item \mbox{[} \mbox{]} Extensible Metadata Platform (X\+MP)
\end{DoxyItemize}

\subsubsection*{Planned features}

\begin{quote}
See Usage (Videos(movies) and images(pictures)) part see more info about audio and video support. \end{quote}

\begin{DoxyItemize}
\item \mbox{[} \mbox{]} Audio formats
\item \mbox{[} \mbox{]} Video formats
\end{DoxyItemize}

\subsection*{Usage}

\subsubsection*{Initialization}

Reference osml library to your app, and execute Initialization with following line\+:

\begin{DoxyVerb}await OSML.Initialization.Run()
\end{DoxyVerb}


This will create main \mbox{\hyperlink{namespaceOSML}{O\+S\+ML}} folder(used for caching and other stuff) and initialize cache(databases, etc...).

Osml does not contain default searching(caching) folders, but you can add some with following line of code\+: \begin{DoxyVerb}OSML.Cache.CacheManager.AddAsync("folder_path_here").Wait();
\end{DoxyVerb}


\begin{quote}
Media database will be automatically updated. \end{quote}


\subsection*{}

\subsubsection*{Getting audio(music)}

After cache is initialized you can get your music using\+: \begin{DoxyVerb}OSML.Data.Music.OSMLAudioData.AllMusic;
\end{DoxyVerb}


\paragraph*{Filters\+:}

\subparagraph*{Filtered by artist name\+:}

\begin{DoxyVerb}OSML.Data.Music.OSMLAudioData.GetFromArtist("artist_name");
\end{DoxyVerb}


\subparagraph*{Filtered by album\+:}

\begin{DoxyVerb}OSML.Data.Music.OSMLAudioData.GetFromAlbum("album_name");
\end{DoxyVerb}


\subparagraph*{Filtered by year\+:}

\begin{DoxyVerb}OSML.Data.Music.OSMLAudioData.GetFromYear("year");
\end{DoxyVerb}


\subsection*{}

\subsubsection*{Videos(movies) and images(pictures)}

Video and image formats are not supported(yet). While those formats are unsupported, osml still caches files and only loads them as Media\+Obj(so no sql(slow)), which means only bare bones data is available. Here is how to get video files\+: \begin{DoxyVerb}foreach(Cache.CacheObj c_obj in Cache.CurrentCData.FolderList) {
    foreach(Media.MediaObj m_obj in c_obj.Media) {
        if(m_obj.Type == Media.MediaType.Video_Movies) {
            System.Console.WriteLine(m_obj.Path);
        }
    }
}
\end{DoxyVerb}


And here is how to get image files(Just change Media.\+Media\+Type.\+Video\+\_\+\+Movies to Media.\+Media\+Type.\+Images\+\_\+\+Pictures) \begin{DoxyVerb}foreach(Cache.CacheObj c_obj in Cache.CurrentCData.FolderList) {
    foreach(Media.MediaObj m_obj in c_obj.Media) {
        if(m_obj.Type == Media.MediaType.Images_Pictures) {
            System.Console.WriteLine(m_obj.Path);
        }
    }
}
\end{DoxyVerb}


\section*{How to contribute}

\subsection*{Prerequisites}


\begin{DoxyItemize}
\item Minimum .N\+ET Core S\+DK version 2.\+2 installed
\item Linux distribution that supports .N\+ET Core(\+Windows is not supported at this moment)
\end{DoxyItemize}

\subsubsection*{Clone repository}

Open terminal and type\+:

git clone \href{https://github.com/NULLCharmander/osml}{\texttt{ https\+://github.\+com/\+N\+U\+L\+L\+Charmander/osml}}

This will clone osml to home directory, or to any other currently opened folder in terminal

\subsubsection*{Tests}

Not available.

\subsubsection*{Pull request}

Create pull request with description about changes 